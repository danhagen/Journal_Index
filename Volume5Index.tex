\documentclass[a4paper]{article} 
\usepackage[english]{babel} 
\usepackage[utf8x]{inputenc} 
\usepackage[T1]{fontenc} 
\usepackage{ragged2e} 
\usepackage{amsmath} 
\usepackage[a4paper,top=3cm,bottom=2cm,left=3cm,right=3cm,marginparwidth=1.75cm]{geometry} 
\begin{document} 
\section*{Index} 
\allowdisplaybreaks 
\begin{flalign*} 
\textit{A\hspace{0.5em}} \\&\text{A Type K Channel} \hspace*{6em}&&p. 30\\
&\text{Absolute Refractory Period} \hspace*{6em}&&pp. 29-30\\
&\text{Acetylcholine} \hspace*{6em}&&pp. 31, 59\\
&\text{Actin} \hspace*{6em}&&pp. 57-58\\
&\text{Action Current} \hspace*{6em}&&p. 55\\
&\text{Activation Gate} \hspace*{6em}&&p. 29\\
&\text{After Potential} \hspace*{6em}&&p. 29\\
&\text{Anodal Surround} \hspace*{6em}&&p. 15\\
&\text{Anode} \hspace*{6em}&&pp. 13, 15\\
&\text{Axial Diameter Or Radius} \hspace*{6em}&&p. 42\\
&\text{Axial Resistance} \hspace*{6em}&&pp. 26-28, 39\\
\textit{B\hspace{0.5em}} \\&\text{Beta Cell } \hspace*{6em}&&p. 38\\
&\text{Biomimetic} \hspace*{6em}&&p. 2\\
&\text{Bionics} \hspace*{6em}&&p. 2\\
\textit{C\hspace{0.5em}} \\&\text{Calcium Activated K Channel} \hspace*{6em}&&p. 30\\
&\text{Capacitance (Neuron)} \hspace*{6em}&&pp. 25-27, 36-37, 40\\
&\text{Capacitive Coupling} \hspace*{6em}&&p. 2\\
&\text{Carrier Frequency} \hspace*{6em}&&pp. 46-47\\
&\text{Cathode} \hspace*{6em}&&pp. 12-13, 15, 17\\
&\text{Channels} \hspace*{6em}&&p. 35\\
&\text{Chronaxie} \hspace*{6em}&&pp. 13, 16\\
&\text{Cochlear Implants} \hspace*{6em}&&p. 4\\
&\text{Concentration Gradient} \hspace*{6em}&&pp. 21, 35-36\\
&\text{Conductance} \hspace*{6em}&&pp. 22, 25, 38\\
&\text{Conduction Velocity} \hspace*{6em}&&pp. 15, 19, 43\\
&\text{Conduction Velocity (Muscle)} \hspace*{6em}&&p. 61\\
&\text{Contraction Time} \hspace*{6em}&&p. 51\\
&\text{Cotransporter} \hspace*{6em}&&p. 23\\
&\text{Cross Talk} \hspace*{6em}&&p. 63\\
&\text{Current (Neuron)} \hspace*{6em}&&pp. 20, 46\\
&\text{Cyborgs} \hspace*{6em}&&p. 2\\
\textit{D\hspace{0.5em}} \\&\text{Deep Brain Stimulation} \hspace*{6em}&&p. 9\\
&\text{Delayed Rectifier} \hspace*{6em}&&p. 30\\
&\text{Dendrocyte} \hspace*{6em}&&p. 43\\
&\text{Depolarization} \hspace*{6em}&&p. 20\\
&\text{Dielectric} \hspace*{6em}&&p. 41\\
&\text{Drop Foot} \hspace*{6em}&&p. 6\\
\textit{E\hspace{0.5em}} \\&\text{Electrochemical Gradient} \hspace*{6em}&&pp. 23, 25-26\\
&\text{Electrogenic} \hspace*{6em}&&p. 23\\
&\text{Electromyography} \hspace*{6em}&&pp. 55-66\\
&\text{Electrotonic Potential} \hspace*{6em}&&p. 20\\
&\text{Epilepsy} \hspace*{6em}&&p. 8\\
&\text{Equilibrium Potential} \hspace*{6em}&&p. 22\\
&\text{Equivalent Circuits} \hspace*{6em}&&pp. 24, 26\\
\textit{F\hspace{0.5em}} \\&\text{Fast Twitch Muscle} \hspace*{6em}&&p. 52\\
&\text{Force Length Curve} \hspace*{6em}&&p. 60\\
&\text{Force Velocity Curve} \hspace*{6em}&&p. 66\\
&\text{Functional Electrical Stimulation} \hspace*{6em}&&pp. 1, 6\\
\textit{G\hspace{0.5em}} \\&\text{Gap Junction} \hspace*{6em}&&p. 38\\
&\text{Goldman Equation} \hspace*{6em}&&pp. 24, 36\\
\textit{H\hspace{0.5em}} \\&\text{Hodgkin Huxley Action Potential Model} \hspace*{6em}&&p. 29\\
&\text{Hyperpolarization} \hspace*{6em}&&pp. 20, 37\\
&\text{Hyperpolization Activated Cation (H.c.n.) Channels} \hspace*{6em}&&pp. 30, 32\\
\textit{I\hspace{0.5em}} \\&\text{Inactivation} \hspace*{6em}&&p. 29\\
&\text{Inactivation Gate} \hspace*{6em}&&p. 29\\
&\text{Innervation Number} \hspace*{6em}&&p. 51\\
&\text{Internode} \hspace*{6em}&&pp. 14, 19-43\\
\textit{M\hspace{0.5em}} \\&\text{M Type K Channel} \hspace*{6em}&&p. 31\\
&\text{Membrane Conductance} \hspace*{6em}&&pp. 26-27, 36, 38\\
&\text{Membrane Resistance} \hspace*{6em}&&pp. 26-28, 39, 41\\
&\text{Motor Unit} \hspace*{6em}&&pp. 51, 63\\
&\text{Multiple Sclerosis} \hspace*{6em}&&pp. 43-44\\
&\text{Multiplexing} \hspace*{6em}&&p. 48\\
&\text{Myelin} \hspace*{6em}&&pp. 28, 32-45\\
&\text{Myelinated Fiber} \hspace*{6em}&&pp. 13, 28-45\\
&\text{Myosin} \hspace*{6em}&&p. 57\\
\textit{N\hspace{0.5em}} \\&\text{Na K Pumps} \hspace*{6em}&&p. 23\\
&\text{Nernst Equation} \hspace*{6em}&&pp. 22, 35\\
&\text{Neural Control} \hspace*{6em}&&p. 1\\
&\text{Neural Prosthetic} \hspace*{6em}&&p. 1\\
&\text{Neuromodulation} \hspace*{6em}&&pp. 1, 8\\
&\text{Neuromuscular Electrical Stimulation} \hspace*{6em}&&p. 1\\
&\text{Node of Ranvier} \hspace*{6em}&&pp. 14, 16\\
&\text{Nyquist Criterion} \hspace*{6em}&&p. 48\\
\textit{P\hspace{0.5em}} \\&\text{Parkinson's Disease} \hspace*{6em}&&p. 8\\
&\text{Passive Response (Neuron)} \hspace*{6em}&&p. 20\\
&\text{Pennation Angle} \hspace*{6em}&&pp. 61-62\\
&\text{Permeability} \hspace*{6em}&&p. 36\\
&\text{Pia Mater} \hspace*{6em}&&p. 14\\
&\text{Plasticity} \hspace*{6em}&&pp. 1, 9\\
&\text{Presynaptic Terminal} \hspace*{6em}&&p. 33\\
&\text{Pyramidal Cell} \hspace*{6em}&&p. 17\\
\textit{R\hspace{0.5em}} \\&\text{Rectification} \hspace*{6em}&&pp. 47, 65-66\\
&\text{Relative Refractory Period} \hspace*{6em}&&p. 30\\
&\text{Resting Membrane Potential} \hspace*{6em}&&p. 20\\
&\text{Rheobase} \hspace*{6em}&&p. 16\\
&\text{Rigor Mortis} \hspace*{6em}&&p. 59\\
\textit{S\hspace{0.5em}} \\&\text{Sampling Frequency} \hspace*{6em}&&p. 48\\
&\text{Sarcomere} \hspace*{6em}&&p. 53\\
&\text{Schwann Cells} \hspace*{6em}&&pp. 43-44\\
&\text{Short Range Stiffness} \hspace*{6em}&&p. 54\\
&\text{Size Principle (Neuron)} \hspace*{6em}&&p. 53\\
&\text{Slow Twitch Muscle} \hspace*{6em}&&p. 52\\
&\text{Space Constant (Neuron)} \hspace*{6em}&&pp. 27-28, 39-40, 45\\
&\text{Spike Train} \hspace*{6em}&&p. 33\\
&\text{Stength Vs Endurance Training} \hspace*{6em}&&p. 52\\
&\text{Strength Distance Relation} \hspace*{6em}&&pp. 12-15\\
&\text{Strength Duration Relationship} \hspace*{6em}&&pp. 16-17, 19\\
\textit{T\hspace{0.5em}} \\&\text{Targeted Muscle Reinnervation} \hspace*{6em}&&p. 7\\
&\text{Tetanus} \hspace*{6em}&&pp. 51, 62\\
&\text{Time Constant (Muscle)} \hspace*{6em}&&pp. 57-58\\
&\text{Time Constant (Neuron)} \hspace*{6em}&&pp. 17, 26-27\\
&\text{Transcutaneous Electrical Nerve Stimulation} \hspace*{6em}&&p. 5\\
&\text{Transverse Tubules} \hspace*{6em}&&pp. 57, 59-60\\
&\text{Trigger Zone (Neuron)} \hspace*{6em}&&p. 32\\
&\text{Trophism} \hspace*{6em}&&pp. 1, 9\\
&\text{Tropomyosin} \hspace*{6em}&&pp. 57-58\\
&\text{Twitch Contraction} \hspace*{6em}&&p. 51\\
&\text{Type I Fibers} \hspace*{6em}&&p. 52\\
&\text{Type Ii Fibers} \hspace*{6em}&&p. 52\\
\textit{V\hspace{0.5em}} \\&\text{Voltage Gated Ca Channel} \hspace*{6em}&&pp. 30-33\\
&\text{Voltage Gated Channels} \hspace*{6em}&&pp. 30-33\\
&\text{Voltage Gated Cl Channel} \hspace*{6em}&&p. 30\\
&\text{Voltage Gated K Channel} \hspace*{6em}&&pp. 30-33\\
&\text{Voltage Gated Na Channel} \hspace*{6em}&&pp. 30-32\\
\end{flalign*} 
\end{document}